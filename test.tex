\documentclass[a4paper, 11pt, dvipdfmx]{jsarticle}
\usepackage[utf8]{inputenc}

\usepackage{tcolorbox}
\tcbuselibrary{breakable, listingsutf8}

\tcbset{
  colframe=black,
  colback=gray!20,
  fonttitle=\bfseries,
  sharp corners = downhill,
  breakable = true
}
\tcbset{
  comandstyle/.style={
    colframe=black,
    colback=black,
    coltitle=white,
    coltext=white,
    fonttitle=\ttfamily,
    sharp corners = all,
    breakable = true,
    title=コマンド,
    boxrule=0.5mm,
    left=1mm,
    right=1mm,
    top=1mm,
    bottom=1mm,
  },
  additionalstyle/.style={
    colframe=red,
    colback=red!10,
    fonttitle=\bfseries,
    rounded corners,
    breakable = true,
    boxrule=0.5mm,
    left=1mm,
    right=1mm,
    top=1mm,
    bottom=1mm
  }
}

\newtcolorbox{commandbox}[2][]{comandstyle, title=#2, #1}

\usepackage{titlesec}
\usepackage{tocloft}

% セクション番号の変更
\titleformat{\section}{\normalfont\Large\bfseries}{\thesection 章}{1em}{}
\titleformat{\subsection}{\normalfont\large\bfseries}{\thesection.\arabic{subsection}}{1em}{}
\titleformat{\subsubsection}{\normalfont\normalsize\bfseries}{\thesection.\arabic{subsection}.\arabic{subsubsection}}{1em}{}

\begin{document}

\title{Linuxサーバー構築手順書}
\author{大阪公立大学工業高等専門学校\\
3年 知能情報コース 5番\\}
\date{\today}
\maketitle\thispagestyle{empty}

\newpage
\setcounter{section}{0}

\section{はじめに}
あいうえお
かきくけこ

さしすせそ
  \begin{commandbox}{}
    \verb|$ sudo apt update|\\
    \verb|$ sudo apt upgrade|
  \end{commandbox}

  \begin{tcolorbox}[additionalstyle]
    あいうえお
  \end{tcolorbox}

\section{Terminal入力の例}
  \begin{commandbox}{Apacheの起動と有効化}
\begin{lstlisting}
sudo systemctl start apache2
sudo systemctl enable apache2
\end{lstlisting}
  \end{commandbox}

  \begin{commandbox}{Apacheの起動}
\begin{lstlisting}
sudo systemctl start apache2
\end{lstlisting}
  \end{commandbox}

\end{document}