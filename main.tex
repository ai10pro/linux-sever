\documentclass[a4paper, 11pt, dvipdfmx]{jsarticle}
\usepackage[utf8]{inputenc}

\usepackage{tcolorbox}
\tcbuselibrary{breakable, listingsutf8}

\tcbset{
  colframe=black,
  colback=gray!20,
  fonttitle=\bfseries,
  sharp corners = downhill,
  breakable = true
}
\tcbset{
  comandstyle/.style={
    colframe=black,
    colback=black,
    coltitle=white,
    coltext=white,
    fonttitle=\ttfamily,
    sharp corners = all,
    breakable = true,
    title=コマンド,
    boxrule=0.5mm,
    left=1mm,
    right=1mm,
    top=1mm,
    bottom=1mm,
  },
  additionalstyle/.style={
    colframe=red,
    colback=red!10,
    fonttitle=\bfseries,
    rounded corners,
    breakable = true,
    boxrule=0.5mm,
    left=1mm,
    right=1mm,
    top=1mm,
    bottom=1mm
  }
}

\newtcolorbox{commandbox}[2][]{comandstyle, title=#2, #1}

\usepackage{titlesec}
\usepackage{tocloft}

% セクション番号の変更
\titleformat{\section}{\normalfont\Large\bfseries}{\thesection 章}{1em}{}
\titleformat{\subsection}{\normalfont\large\bfseries}{\thesection.\arabic{subsection}}{1em}{}
\titleformat{\subsubsection}{\normalfont\normalsize\bfseries}{\thesection.\arabic{subsection}.\arabic{subsubsection}}{1em}{}

\begin{document}

\title{Linuxサーバー構築手順書}
\author{大阪公立大学工業高等専門学校\\
3年 知能情報コース 5番\\
}
\date{\today}
\maketitle\thispagestyle{empty}

\newpage
\setcounter{section}{-1}

\section{はじめに}

\section{Ubuntuインストール}
\subsection{ディストリービューしょんの準備}
\subsection{Boot Modeの変更}
  \begin{enumerate}
    \item 電源をつけF2を連打しBIOS画面を表示
    \item Securityに移動しSupervisor Passwordを設定
    \item Bootに移動しBoot ModeをLegacyに変更
    \item F10で保存して再起動
    \item F2を連打しBIOS画面を表示
    \item Bootに移動しBoot Priorityを変更しインストールメディアが入るものを選択
    \item Boot ModeをUEFIに変更
    \item SecureBootをDisableに変更
    \item F10で保存して再起動
  \end{enumerate}
\subsection{Ubuntuのインストール}
  \begin{enumerate}
    \item 再起動後自動的にBoot Menuが表示される
    \item Try or Install Ubuntuを選択
    % \item 以下Ubuntuのインストールを書く ★
    \item インストールの種類を選択
    \item インストール先を選択
    \item キーボードレイアウトを選択
    \item ユーザー情報を入力
    \item インストールが完了したら再起動
  \end{enumerate}
\subsection{インストール後の再起動}
  \begin{enumerate}
    \item 再起動時にBiosに入る
    \item Bootに移動しBoot Priorityを変更しインストールしたディスクを1番目にする
    \item SecureをEnableに変更
    \item Securityに移動しErase all Secure Boot Settingを選択し、Boot設定を削除
    \item Select an UEFI file as trusted for executingを選択
    \item EMMC > EFI > Ubuntu > shimx64.efiを選択
    \item Boot名を入力し、Enter
    \item Bootに移動しSecure BootをDisableに変更
    \item Securityに移動しSupervisor Passwordを削除
    \item F10で保存して再起動
  \end{enumerate}

\end{document}